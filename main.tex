\documentclass{article}
\usepackage[utf8]{inputenc}
\usepackage{amssymb,amsmath,graphicx,indentfirst}

\title{EP5 - FBST}
\author{George Othon - NUSP 103xxxxx 
        \\ Felipe Zaffalon - NUSP 103xxxxx}
\date{Junho de 2020}

\begin{document}

\maketitle

\section{Introdução}
Nesse EP deveriamos implementar um teste de Hipótese apresentado no artigo "Evidence and
Credibility: Full Bayesian Signicance Test for Precise Hypotheses", para isso deveriamos reproduzir os resultados do e-valor da hypothese de Hardy-Weinberg para os resultados observacionais
constantes na Tabela 2 do Artigo.

\section{Função e hipótese nula}
Para testar a Hipótese Nula precisariamos primeiro otimizar a função:

$$ f(\theta_1, \theta_2, \theta_3, x1, x2, x3) = \theta_1^{x1}.\theta_2^{x2}.\theta_3^{x3}
$$

\hfill

\par Com as condições de $1 = \theta_1+\theta_2+\theta_3, \theta_1, \theta_2$, $\theta_3 > 0$ , e a condição da hipótese de $\theta_3$ = (1 - $ \sqrt\theta_1)^2$,
com x1 e x3 tabelos e x2 = n - x1 - x3 com n = 20. Para isso escrevemos a função f em torno de $\theta_1$
e um índice i para definir os x1, x2 e x3 necessarios para os parametros, na otimização utilizamos a função optimise() do r, com maximum = TRUE para receber o valor do máximo da função, com isso temos dois vetores de resultados, um deles com os $\theta_{max}$ para cada índice de x1,x2,x3, e
um com os $f_{max}$ para cada um desses $\theta_{max}$.

\section{Função de integração por MCMC}
Nessa parte precisavamos calcular o e-valor apresentado pela função em cada um dos trios usando
o metódo de integração por MCMC.
Para a segunda parte do ep, a integração por MCMC, escrevemos uma função que recebe os x1 e
x3 tabelados e os valores s1 e s3, sendo esses a variância usada para o cálculo da normal. Num
primeiro momento geramos $\theta_1, \theta_3$ aleatórios, e $\theta_2$ em função de ambos, sendo assim o trio ($\theta_1,$ $\theta_2$, $\theta_3$)
sempre respeitasse as condições ditas anteriormente, depois disso ocorria o cálculo
da normal e se gerava mais um trio de $\theta$'s, e comparando ambos os trios se decidia qual deles
entraria para a lista retornada usando o aplha de aceitação de metrópolis, após a finalização da
lista comparavamos agora com os valores obtidos após a otimização no primeiro momento, depois
calculávamos a média dessa iteração nos devolvendo o objetivo.

\section{Calibragem do parâmetro da normal}
Por muitos testes empíricos e gerações com multiplos parametros decidimos que um valor aceitável era
o valor de 1.

\section{Escolha de passos}
Para uma boa precisão fizemos 5 iterações utilizando a média dos 5 como resultado para cálculo do e-valor,
sem necessidade de burn-in, os resultados ao final de todo o processo se mostraram consistentemente próximos.

\section{Método de Metropolis}
Para a escolha de $\alpha$, usamos o Método de Metropolis dado por:
$$\alpha = min( 1, \frac{g(\theta_{1proposto}, \theta_{3proposto}, índice)}{g(\theta_{1atual}, \theta_{3atual}, índice)}})$$

sendo indice referente aos valores da tabela para x1 e x3.



\end{document}
